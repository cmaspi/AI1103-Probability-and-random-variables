\documentclass[journal,12pt,twocolumn]{IEEEtran}

\usepackage{setspace}
\usepackage{gensymb}
\singlespacing
\usepackage[cmex10]{amsmath}

\usepackage{amsthm}

\usepackage{mathrsfs}
\usepackage{txfonts}
\usepackage{stfloats}
\usepackage{bm}
\usepackage{cite}
\usepackage{cases}
\usepackage{subfig}

\usepackage{longtable}
\usepackage{multirow}

\usepackage{enumitem}
\usepackage{mathtools}
\usepackage{steinmetz}
\usepackage{tikz}
\usepackage{circuitikz}
\usepackage{verbatim}
\usepackage{tfrupee}
\usepackage[breaklinks=true]{hyperref}
\usepackage{graphicx}
\usepackage{tkz-euclide}

\usetikzlibrary{calc,math}
\usepackage{listings}
    \usepackage{color}                                            %%
    \usepackage{array}                                            %%
    \usepackage{longtable}                                        %%
    \usepackage{calc}                                             %%
    \usepackage{multirow}                                         %%
    \usepackage{hhline}                                           %%
    \usepackage{ifthen}                                           %%
    \usepackage{lscape}     
\usepackage{multicol}
\usepackage{chngcntr}

\DeclareMathOperator*{\Res}{Res}

\renewcommand\thesection{\arabic{section}}
\renewcommand\thesubsection{\thesection.\arabic{subsection}}
\renewcommand\thesubsubsection{\thesubsection.\arabic{subsubsection}}

\renewcommand\thesectiondis{\arabic{section}}
\renewcommand\thesubsectiondis{\thesectiondis.\arabic{subsection}}
\renewcommand\thesubsubsectiondis{\thesubsectiondis.\arabic{subsubsection}}


\hyphenation{op-tical net-works semi-conduc-tor}
\def\inputGnumericTable{}                                 %%

\lstset{
%language=C,
frame=single, 
breaklines=true,
columns=fullflexible
}
\begin{document}

\newcommand{\BEQA}{\begin{eqnarray}}
\newcommand{\EEQA}{\end{eqnarray}}
\newcommand{\define}{\stackrel{\triangle}{=}}
\bibliographystyle{IEEEtran}
\raggedbottom
\setlength{\parindent}{0pt}
\providecommand{\mbf}{\mathbf}
\providecommand{\pr}[1]{\ensuremath{\Pr\left(#1\right)}}
\providecommand{\qfunc}[1]{\ensuremath{Q\left(#1\right)}}
\providecommand{\sbrak}[1]{\ensuremath{{}\left[#1\right]}}
\providecommand{\lsbrak}[1]{\ensuremath{{}\left[#1\right.}}
\providecommand{\rsbrak}[1]{\ensuremath{{}\left.#1\right]}}
\providecommand{\brak}[1]{\ensuremath{\left(#1\right)}}
\providecommand{\lbrak}[1]{\ensuremath{\left(#1\right.}}
\providecommand{\rbrak}[1]{\ensuremath{\left.#1\right)}}
\providecommand{\cbrak}[1]{\ensuremath{\left\{#1\right\}}}
\providecommand{\lcbrak}[1]{\ensuremath{\left\{#1\right.}}
\providecommand{\rcbrak}[1]{\ensuremath{\left.#1\right\}}}
\theoremstyle{remark}
\newtheorem{rem}{Remark}
\newcommand{\sgn}{\mathop{\mathrm{sgn}}}
\providecommand{\abs}[1]{\vert#1\vert}
\providecommand{\res}[1]{\Res\displaylimits_{#1}} 
\providecommand{\norm}[1]{\lVert#1\rVert}
%\providecommand{\norm}[1]{\lVert#1\rVert}
\providecommand{\mtx}[1]{\mathbf{#1}}
\providecommand{\mean}[1]{E[ #1 ]}
\providecommand{\fourier}{\overset{\mathcal{F}}{ \rightleftharpoons}}
%\providecommand{\hilbert}{\overset{\mathcal{H}}{ \rightleftharpoons}}
\providecommand{\system}{\overset{\mathcal{H}}{ \longleftrightarrow}}
	%\newcommand{\solution}[2]{\textbf{Solution:}{#1}}
\newcommand{\solution}{\noindent \textbf{Solution: }}
\newcommand{\cosec}{\,\text{cosec}\,}
\providecommand{\dec}[2]{\ensuremath{\overset{#1}{\underset{#2}{\gtrless}}}}
\newcommand{\myvec}[1]{\ensuremath{\begin{pmatrix}#1\end{pmatrix}}}
\newcommand{\mydet}[1]{\ensuremath{\begin{vmatrix}#1\end{vmatrix}}}
\numberwithin{equation}{subsection}
\makeatletter
\@addtoreset{figure}{problem}
\makeatother
\let\StandardTheFigure\thefigure
\let\vec\mathbf
\renewcommand{\thefigure}{\theproblem}
\def\putbox#1#2#3{\makebox[0in][l]{\makebox[#1][l]{}\raisebox{\baselineskip}[0in][0in]{\raisebox{#2}[0in][0in]{#3}}}}
     \def\rightbox#1{\makebox[0in][r]{#1}}
     \def\centbox#1{\makebox[0in]{#1}}
     \def\topbox#1{\raisebox{-\baselineskip}[0in][0in]{#1}}
     \def\midbox#1{\raisebox{-0.5\baselineskip}[0in][0in]{#1}}
\vspace{3cm}
\title{Assignment 1}
\author{Chirag Mehta - AI20BTECH11006}
\maketitle
\newpage
\bigskip
\renewcommand{\thefigure}{\theenumi}
\renewcommand{\thetable}{\theenumi}
Download all python codes from 
\begin{lstlisting}
https://github.com/cmapsi/AI1103-Probability-and-random-variables/tree/main/Assignment-1/codes
\end{lstlisting}
and latex-tikz codes from 
\begin{lstlisting}
https://github.com/cmapsi/AI1103-Probability-and-random-variables/blob/main/Assignment-1/main.tex
\end{lstlisting}
\section{Problem}
Complete the following statements:\\
(i) Probability of an event E + Probability of the event ‘not E’ =———– .\\
(ii) The probability of an event that cannot happen is———- . Such an event is called— —— . \\
(iii) The probability of an event that is certain to happen is ———. \\
(iv) The sum of the probabilities of all the elementary events of an experiment is———-.\\
(v) The probability of an event is greater than or equal to ————– and less than or equal to ————–.

\section{Solution}
\subsection{}
 It can be noted that the two events 'E' and 'not E' are disjoint because the event 'E' can either happen or not happen.
By axiom of probability theory 
\begin{align}
	Pr\brak{E}+Pr\brak{E'}=Pr\brak{E+ E'}
\end{align} 
which is event E may nor may not happen. Since 'E' and 'not E' are exhaustive events, 
\begin{align}
	Pr\brak{E+ E'}=1.
\end{align}
\subsection{}
The probability of an event that cannot happen is \textbf{0}. Such an event is called \textbf{impossible event}\\
In crude sense, probability is the ratio of favourable cases to total number of cases. Since the event cannot occur, the number of favourable cases will always be 0 for any number of total cases in sample. Since this definition is consistent with actual definition of probabilty over suffieciently large sample size, we can say that the probabilty of such an event will be 0.
\subsection{}
The probabilty of an event that is certain to happen is \textbf{1}.
Similar to previous part, every case in sample will also be a favourable case. Therefore, the probability of such an event is 1.
\subsection{}
Elementary events are disjoint ($\because \{x\}\cap\{y\}=\phi$, if $x\neq y$ (for singleton set x and y)).\\
By axiom of probability 
\begin{align}
	Pr\brak{E_1}+Pr\brak{E_2}+...+Pr\brak{E_n}=Pr\brak{E_1+ E_2+ ...+ E_n}
\end{align}
Where $E1,\ E_2,...,\ E_n$ are the elementary events.\\
By definition of elementary events, they must be exhaustive.
\begin{align}
	E_1\cup E_2\cup ...\cup E_n=S
\end{align}
Where S is the sample space. By axiom of probability theory, $Pr\brak{S}=Pr\brak{E_1+ E_2+ ...+ E_n}=1$.
\subsection{}
The probability of an event is greater than or equal to \textbf{0}, and less than or equal to \textbf{1}.
By axiom of probabilty, $Pr\brak{E}\geq 0$.\\
We know that for any event E, $E\subseteq S$, where S is the sample space. Since $P(S)=1$ (axiom), $Pr\brak{E}\leq Pr\brak{S}=1$.\\
\end{document}
